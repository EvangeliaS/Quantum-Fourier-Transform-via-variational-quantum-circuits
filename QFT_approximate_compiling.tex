\documentclass[pra,onecolumn,superscriptaddress]{revtex4}%
\usepackage{amssymb}
\usepackage{amsmath}
\usepackage{graphicx}
\usepackage{dcolumn}
\usepackage{xcolor}
\usepackage{bm}
\usepackage{subfigure}
\usepackage{amsfonts}
\usepackage{appendix}
\usepackage{xcolor}%
\setcounter{MaxMatrixCols}{30}
\providecommand{\U}[1]{\protect\rule{.1in}{.1in}}
%EndMSIPreambleData
\def\e{\varepsilon}

\renewcommand{\Im}{{\rm Im}}
\newtheorem{theorem}{Theorem}[section]
\newtheorem{lemma}[theorem]{Lemma}
\newtheorem{proposition}[theorem]{Proposition}
\newtheorem{corollary}[theorem]{Corollary}
\newenvironment{proof}[1][Proof]{\begin{trivlist}
\item[\hskip \labelsep {\bfseries #1}]}{\end{trivlist}}
\newenvironment{definition}[1][Definition]{\begin{trivlist}
\item[\hskip \labelsep {\bfseries #1}]}{\end{trivlist}}
\newenvironment{example}[1][Example]{\begin{trivlist}
\item[\hskip \labelsep {\bfseries #1}]}{\end{trivlist}}
\newenvironment{remark}[1][Remark]{\begin{trivlist}
\item[\hskip \labelsep {\bfseries #1}]}{\end{trivlist}}

\begin{document}
\preprint{APS/123-QED}
\title{ Quantum Fourier Transform with a variational circuit }
\author{...Katerina..}
\affiliation{Department of Informatics and Telecommunications, National and Kapodistrian
University of Athens, Panepistimiopolis, Ilisia, 15784, Greece}


\begin{abstract}
 The aim of this project is to achieve quantum compiling for QFT unitary operation
employing a subset of physical interactions between qubits and a circuit
of low-depth. 

\end{abstract}
\maketitle

\section{Introduction}
QFT is an essential operation in quantum computing. To be realized for $n$ qubits this requires 
$\approx n^2$ gates that is an excellent record --taking into account that a general unitary operation
is described by $4^n$ real parameters. On the other hand, this includes controlled-phase gates $R_n$
each of which if to be decomposed into gates from a universal set, requires with the best known
method up today $\approx 20$ gates and an ancillary qubit. 

In the NIST era the idea of employing fault-tolerant quantum computation has been set aside
since deep-depth circuits are out of discussion. In this work we investigate the question
of realizing the QFT operation with a variational circuit of relatively low depth. Instead of
working with universal set of gates we employ Hamiltonians which can be in principle physically realizable.
 The initial procedure that we follow for a $3$-qubit is a standard one:
\begin{itemize}
	\item We build an ansantz on the structure of the circuit based on the algebraic properties of QFT.
	\item We define a cost function that measures the distance from the target operation i.e. QFT.
	\item We optimize the parameters in the circuit with a stochastic gradient descent so that the cost is minimized. 
\end{itemize}
At second stage, we minimize as possible the depth of the circuit and we update the cost function to an experimentally accessible one.
Naturally in order to arrive to useful results we try to generalize working with $4$ and $5$ qubit circuits.
  
\section{QFT on $3$-qubits} 


In order to build an ansatz we first take the unitary matrix (in the computational basis), $\hat{U}_{QFT}$,
and we decompose it onto the $64$ generators of $SU(8)$ algebra, i.e. $\hat{g}_{ijk}=\hat{\sigma}_{i} \otimes \hat{\sigma}_{j} \otimes \hat{\sigma}_{k}$,
as 
\begin{equation}
O_{ijk}=Tr\left(\hat{U}_{QFT} \hat{g}_{ijk} \right).
\end{equation}
The $\hat{\sigma}_{i}$ with $i=1,2,3$ are the Pauli matrices for a single qubit while $\hat{\sigma}_{4}=\hat{1}$.

We find out (see mathematica file) that only $20$ generators give non-zero overlap $O_{ijk}$.

We commute these $20$ generators and we find out that these together with first, second,.. order commutations form
a closed subgroup of $32$ elements. Then we search for a minimum set of single-qubit and two-qubit operators able to generate the whole
subgroup. We identify $11$ of them ($5$ single qubit and $6$ two-qubit Hamiltonians) but possibly this result could be improved.

We build a parametrized circuit with these $N=11$ operators, e.g. $\hat{U}_{ijk}\left(\phi\right)=\left(-I \phi \hat{g}_{ijk}\right)$ and we search to optimize the $11$ angles $\vec{\phi}=\left\{\phi_1,\ldots, \phi_{11}\right\}$ so that the cost function is as low as possible:
\begin{equation}
C\left(\hat{U}_c\left(\vec{\phi}\right), \hat{U}_{QFT}\right)=1-\frac{1}{64}\left|Tr\left(\hat{U}_c\left(\vec{\phi}\right)\hat{U}_{QFT}^{\dagger} \right)\right|^2
\end{equation}
where $\hat{U}_c\left(\vec{\phi}\right)$ represents the unitary of the variational circuit.

The results of some preliminary tests:
\begin{itemize}
	\item $N=11$, minimum $C=0.6$
	\item $N=22$, minimum $C=0.038$
	\item $N=29$, minimum $C=0.00005$.
\end{itemize}
From these we conclude that for a good approximation the depth cannot be as low as desired.
On the other hand when I replace the two-qubit operations with random ones (from the set of $64$ generators) the 
results much deteriorate which exhibits the fact that the ansatz is strong. For instance I get for $N=29$, minimum $C=0.074$.

Much left to be done and to be investigated but at first look there is some promise, having double two-qubit operations
than the initial circuit but the flexibility to work with physical operations than rigid CNOT gates.
The essence of the method will be in the scaling.


\begin{thebibliography}{99}                                                                                               %




\end{thebibliography}



\end{document}